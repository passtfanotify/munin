\documentclass{beamer}

\usepackage[ngerman]{babel}


\author{Bernd Bassimir \& Nico D\"orr}
\title{Abschluss Praktikum}
\date[20.9.2013]{20. September 2013}

\usetheme{Copenhagen}
\beamertemplatenavigationsymbolsempty
\useoutertheme{default}

\begin{document}
\begin{frame}
  \maketitle
\end{frame}
\begin{frame}{R\"uckblick: Unsere Ziele}
  \begin{itemize}
  \item
    Unabh\"angiger Daemon auf fanotify-Basis zur \"Uberwachung von
    Datei\"anderungen \checkmark
  \item
    \"Uberarbeitete Backup-Software mit Schnittstelle zum Daemon
  \item
    Funktionierendes Backupsystem, womit alle zu sichernden Dateien erfasst
    werden \checkmark
  \item
    Beschleunigung des Backup-Vorgangs im Vergleich zur urspr\"unglichen Version
    von rdiff-Backup \checkmark
  \item
    Gespr\"ache mit der Linux-Community und den rdiff-Backup-Entwicklern
    bez\"uglich der Verwendung unserer L\"osung
  \end{itemize}
\end{frame}

\begin{frame}{Fanotify Daemon}
  Erfolge:
  \begin{itemize}
    \item
      Daemon ist konfigurierbar
    \item
      Dateisystem ist mittels fanotify \"uberwacht
    \item
      Einschr\"ankung des Dateisystembaums mittels whitelist
    \item
      Output nebenl\"aufig, um keine Datei\"anderungen zu verlieren
  \end{itemize}
\end{frame}

\begin{frame}{Fanotify Daemon}
  Aufgetretene Probleme:
  \begin{itemize}
    \item
      Starke Einschr\"ankung von fanotify:
      \begin{itemize}
      \item
        Entweder nur ganze Mount-Points markierbar oder nur einzelne Ordner ohne
        Unterordner
      \item
        Blacklisting nicht mittels fanotify m\"oglich
      \item
        L\"oschoperationen werden nicht registriert
      \end{itemize}
    \item
      Schwierige Kommunikation mit den verschiedenen Usern
  \end{itemize}
\end{frame}

\begin{frame}{rdiff-Backup}
  \begin{itemize}
    \item
      Code nicht gen\"ugend dokumentiert
    \item
      Keine Reaktion auf die Anfrage bez\"uglich der Erweiterung mittels des
      Daemons
    \item
      Einf\"ugen der eigenen Schnittstelle stellte sich als schwierig heraus
    \item
      Versuch aus Zeitgr\"unden aufgegeben
  \end{itemize}
\end{frame}

\begin{frame}{Tests}
\begin{itemize}
  \item
    /usr/src/ als Testverzeichnis benutzt
  \item
    ca. 140k Dateien im Verzeichnis vorhanden
  \item
    Getestet wurde mit touch. Dabei wurden 1\%/5\%/10\%/15\%/20\% angefasst.
  \item
    Wegen Zeiteinschr\"ankungen waren weitere Tests nicht mehr m\"oglich
\end{itemize}
\end{frame}

\begin{frame}
  Lustige plots: Under construction
\end{frame}

\begin{frame}
  Lustige plots: Under construction
\end{frame}

\begin{frame}{Zuk\"unftige Anpassungen}
  \begin{itemize}
    \item
      Verbesserung der Kommunikation Daemons mit den verschiedenen Usern
    \item
      Erweiterung des Daemons zur Benutzung von Blacklists
    \item
      Verbesserung von fanotify zum blacklisten und whitelisten
  \end{itemize}
\end{frame}

\begin{frame}
  \begin{center}
  \huge{Danke f\"ur die Aufmerksamkeit!}\\
  \huge{Noch Fragen?}
\end{center}
\end{frame}

\end{document}